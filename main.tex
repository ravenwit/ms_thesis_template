
\documentclass[master, copyright]{resources/class/ms_thesis}


%% Load your required packages here


\usepackage{url}
\usepackage{amsmath}
\usepackage{mathenv}
\usepackage{amssymb}
\usepackage{amsthm}
\usepackage{textcomp}
\usepackage{longtable}
\usepackage{multirow}
\usepackage{pifont}
\usepackage{changepage}
\usepackage{listings}
\usepackage{url}
\usepackage{xspace}
\usepackage{xtab}
\usepackage[utf8]{inputenc}
\usepackage[T1]{fontenc}
\usepackage{graphicx}


%%%%% Preamble: Thesis Particulars %%%%%%%%%%%%%%%%%%%%%%%%%%%%%%%%%
%%%%%%%%%%%%%%%%%%%%%%%%%%%%%%%%%%%%%%%%%%%%%%%%%%%%%%%%%%%%%%%%%%%%%%


\title{The Large Numbers Hypothesis and the rotation of the Earth}
\author{Author Name}
\submissionDate{26}{February}{2017}
\keywords{Diarc, Large number, Dimensionless constant}
\degree{Master of Science}      


\degreeHeld{B.Sc (Honors)}
\dept{Department of Physics}
\school{School of Physical Sciences}


\dedicationtrue    % Not required.
\dedication{Someone special \\or/and\\ Dedicated people}


%%%%%%%%%%%%%%%%%%%%%%%%%%%%%%%%%%%%%%%%%%%%%%%%%%%%%%%%%%%%%%%%%%%%%%%%%

%\noListOfTables  
%\noListOfFigures   

\begin{document}


%% Must be placed before \abstract and \acknowledgements
\frontmatter  
%


%%%%%%%%%%%%%%%%%%%%%%%%%%%% Abstract %%%%%%%%%%%%%%%%%%%%%%%%%%%%%%%%%


\begin{abstract}      
My abstract goes here.
\end{abstract}

%%%%%%%%%%%%%%%%%%%%%%%%%%%%%%%%%%%%%%%%%%%%%%%%%%%%%%%%%%%%%%%%%%%%%%%%


%%%%%%%%%%%%%%%%%%%%%%%%%%%% Acknowledgement %%%%%%%%%%%%%%%%%%%%%%%%%%%%%%%%%


\begin{acknowledgements}
I would like to thank ...
\end{acknowledgements}

%%%%%%%%%%%%%%%%%%%%%%%%%%%%%%%%%%%%%%%%%%%%%%%%%%%%%%%%%%%%%%%%%%%%%%

\mainmatter


%%%%%%%%%%%%%%%%%%%%%%%%%%%% Chapters %%%%%%%%%%%%%%%%%%%%%%%%%%%%%%%%%
\chapter{Introduction}
This is my chapter 1 Intorduction.
\section{Overview}
This is my Overview section
\newpage
\section{Another Section}
This is another section on a new page to show the header.
\chapter{Some predictions of the Large Numbers Hypothesis}
The variations with cosmic time predicted by the LNH of the Earth's orbital radius. its mean
angular motion  and other parameters relevent to the discussion of Section 4 are summarized in Table 1 which I have adapted from articles by Van Flandern (1976) and Faulner(1976). Before proceeding, a brief explanation of the entries of the Table 1 needed. 
\newline \\
For Dirac's early LNH the variation of the orbital elements has beeen teken from the discussion of Vinti. In essence, these relationships amount to the assumption tat G varies at $t^{-1}$ and that angular momentum is conserved. 
%chapter{More Chapters}
\chapter{The interpretation of the fossil growth patterns}
Let $n_{E}$ be the rotational velocity of the Earth, $n_{M}$ the mean angular momentum of the Moon about the Earth and $n_{S}$ the mean motion of the Earth about the Sub. For the purpose of the analysis of Section 4 it is convenient to express the past variation on the rotation of the Earth in terms of the numbers of sidereal days in the sidereal year $(Y=n_{E}/n_{S})$ and the sidereal month $(M=n_{E}/n_{M})$.
\newline \\
It is ususally assumed that the biological growth patterns found in fossil bivale mollusces preserves a record of the numbers of solar days in the tropical  year $(Y=(n_{E}-n_{S})/n_{S})$, ignoring the small difference between the tropical and sidereal years) and the synodic month $[M'=(n_{E}-n_{S})/(n_{M}-n_{S})]$.
\newline \\
\chapter{The application of the tidal theory}

\chapter{The variation of the Earth's moment of inertia}

\chapter{Discussion and conclusion}

%%%%%%%%%%%%%%%%%%%%%%%%%%%%%%%%%%%%%%%%%%%%%%%%%%%%%%%%%%%%%%%%%%%%%%

%% Appendices       
\appendix

\chapter{\LaTeX\ Resources}
This is appendix A \cite{ref01}.
\newline \\
Another reference \cite{ref02}
\begin{center}
  
\end{center}

\bibliographystyle{plain}
\bibliography{resources/bibliography/bib}
\end{document}
